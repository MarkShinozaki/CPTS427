\documentclass{article}
\usepackage{amsmath}

\begin{document}

\title{Cpts 427 Homework 2}
\author{Mark Shinozaki}
\date{February 29, 2024}
\maketitle

\section{Chapter 1}
Select three of the problems with wireless from page 7. Explain each problem in more detail, as well as how each problem is addressed.

\subsection{Multiple-input Multiple-output}
Multiple-input Multiple-output or MIMO is a method for multiplying the capacity of a radio link using multiple transmission and receiving antennas to exploit multipath propagation. MIMO systems require careful design to achieve their potential benefits. The most significant problem is the interference and signal fading caused by multipath propagation. Multiple antennas at both the transmitter and receiver, MIMO systems take advantage of the different signal paths to improve the quality and speed of data transmission.

\subsection{Error control coding}
Error control coding is used to improve wireless communication by detecting and correcting errors in that communication that occur during the transmission of data. Using a technique called Golay Code, it has properties that can be utilized to devise a parallel decoder that monitors errors. The main problem addressed by error control coding is the errors that can occur when data transmission occurs, this may be caused by noise or for example interference or maybe other factors.

\subsection{Direct sequence spread spectrum}
Direct sequence Spread Spectrum or DSSS is a spread-spectrum modulation technique primarily used to reduce overall signal interference. The direct sequence modulation makes the signal wider within bandwidth than information bandwidth. After the dispreading or removal of direct sequence modulation in the receiver the information bandwidth is restored, this is while unintentional interference is reduced. The main problem addressed by DSSS is the interference that can occur when multiple signals are transmitted over the same frequency band. This means by spreading the signal across a wide frequency band, DSSS can reduce the impact of the interference.

\section{Chapter 2}
\subsection{Define bandwidth}
Bandwidth refers to the maximum rate of data transfer across a given path. It is also the range within a band of wavelengths, frequencies, or energies, especially a range of radio frequencies which is occupied by a modulated carrier wave. Also bandwidth can be summarized as data transfer capacity of a network in bits per second or Bps.

\subsection{Compare and contrast analog signals, digital signals, analog data and digital data}
\begin{itemize}
\item Analog Signals: are continuous signals which represent physical measurements. They are represented by sine waves and use continuous range of values to represent information.
\item Digital Signals: are discrete time signals generated by digital modulation. This includes square waves and use discrete or discontinuous values to represent information.
\item Analog data: is meant to represent data in a physical, or continuous form. For example, the human voice in air or the voltage in an electronic device are forms of analog data.
\item Digital Data: This is data that is represented in a binary digital form. It consists of discrete values for example 0s and 1s. Data stored in a computer or transmitted over a digital network is digital data.
\end{itemize}

\subsection{What is the relationship between wavelength and frequency?}
The relationship wavelength and frequency is inversely proportional. As the frequency of a wavelength increases, the wavelength decreases.

\subsection{Compare and contrast guided media and unguided media}
\begin{itemize}
\item Guided media is also known as wired communication or bounded transmission media, this involves the physical pathways like cables for data transmission. It provides, reliability security and high bandwidth, making it ideal for many scenarios. A crucial feature is It requires physical access to the network.
\item Unguided media, is also known as, wireless communication or unbounded transmission media, it allows data transmission without the use of physical conduits. Since it transmits signals openly through the air, it is more susceptible to eavesdropping and requires more robust encryption and protocols.
\end{itemize}

\subsection{Solve problems 2.1}
The period of the signal would be 0.001 signal. Period (T) = 1 / Frequency (f), T = 1 / 1000 = 0.001 seconds

\subsection{Solve problems 2.9}
Bandwidth, the channel has a bandwidth of 300Hz, SNR or signal to Noise ratio is 3 dB. SNR (linear) = $10^{(SNR (dB) / 10)} = 10^{(3 / 10)} = 2$, Capacity = B * $log_2(1 + SNR)$, Capacity = 300 * $log_2(1 + 2) = 300 * log_2(3) \approx 300 * 1.585 = 475.5$ bps

\subsection{Solve problems 2.11}
The relationship between both theorems addressed is channel capacity, but Nyquist focuses on noiseless channels, while Shannon’s theorem accounts for noise. Shannon’s capacity is a generalization of Nyquist's result, incorporating the impact of noise on communication systems. Nyquist and Shannon provide complementary insights into channel capacity, with Nyquist laying the foundation and Shannon extending it to more noisy channels.

\section{Chapter 3}
\subsection{Address problems 3.7}
The flaw in this reasoning is that it overlooks the efficiency gained from packet switching’s ability to share a network paths among multiple users dynamically, this can lead to higher overall network utilization. While packet switching introduces overhead with control and address bits, its ability to efficiently manage network resources lead to better line utilization when compared to fixed paths.

\section{Chapter 4}
\subsection{Answer review question 4.3, “What is a protocol?”}
A protocol is a set of rules for formatting and processing data. Network protocols are like a common language for computers. Computers within a network may use vastly different software and hardware, however, the use of protocols enables them to communicate with each other regardless.

\subsection{Solve problem 4.8}
(math down on paper) To fit the data into the destination networks maximum packet size, it will be fragmented into 3 packets. Therefore, the total number of bits delivered to the network layer protocol at the destination, including all headers, is 1892 bits.

\section{Chapter 5}
\subsection{Answer review question 5.12, “why would you expect a CRC to detect more errors than a parity bit?”}
Cyclic Redundancy Check is expected to detect more errors than a parity bit because it provides more redundancy. CRC provides more information that can be used to detect errors. CRC is more robust in error detections compared to a parity bit.

\subsection{Solve problem 5.2}
Directional characteristics, Indicates the antennas ability to radiate or receive energy in different directions. Beamwidth, is the angular width between the points on the radiation pattern at which the signal strength drops to half its maximum value, which is typically measured in degrees, Gain, is the measure of how much power is radiated in the direction of strongest radiation compared to a reference antenna, Polarization, shows the orientation of the electric field of the radiated wave, which can be linear, circular, or elliptical. Then lastly, Front to back ratio, is the ratio of power radiated in the main lobe to the power radiated in the opposite direction.

\subsection{Solve problem 5.4}
Difference between Diffraction and scattering,
\begin{itemize}
\item While diffraction deals with the bending of waves around obstacles or through apertures, scattering involves the redirection of waves due to interactions with small particles or irregularities.
\end{itemize}

\end{document}
